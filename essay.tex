% Please do not change the document class
\documentclass{scrartcl}

% Please do not change these packages
\usepackage[hidelinks]{hyperref}
\usepackage[none]{hyphenat}
\usepackage{setspace}
\doublespace

% You may add additional packages here
\usepackage{amsmath}

% Please include a clear, concise, and descriptive title
\title{What actions Can be taken to correctly critique design choices in the development cycle?}

% Please do not change the subtitle
\subtitle{COMP150 - Agile Development Practice}

% Please put your student number in the author field
\author{1707502}

\begin{document}

\maketitle

\abstract{This question will be answered by looking at different sources that discuss communication and issues with communication within the development cycle. We will also find whether communication issues are more common in agile development when compared to other forms of development cycles (such as waterfall) because of the frequency of meetings and the high level of attendance from each member of the team that this requires. We will also look at how we can fix and prevent situations in which specifically two members of a team strongly disagree on a design choice and how this can have a affect on the team working with them. I will also consider looking into personality traits and how keeping them in mind can help the work process.}

\section{Introduction}

This question will be answered by looking at different sources that discuss communication and issues with communication within the development cycle. We will also find whether communication issues are more common in agile development when compared to other forms of development cycles (such as waterfall) because of the frequency of meetings and the high level of attendance from each member of the team that this requires. We will also look at how we can fix and prevent situations in which specifically two members of a team strongly disagree on a design choice and how this can have a affect on the team working with them. I will also consider looking into personality traits and how keeping them in mind can help the work process.

\section{Your section title here}

Write the main body of your essay here. Add more sections if appropriate. You may choose to write about each of your three papers in its own section, or you may choose a different structure. Either way, remember that you are being assessed on technical insight and analysis: it is not enough to merely summarise the contents of the three papers. You must demonstrate the ability to make inferences beyond what is written in the papers, and to draw the three papers together into a single coherent narrative.

Your essay must make a clear recommendation, in terms of which of the three techniques you have reviewed is the best according to whichever metric or metrics you feel is most appropriate. You must justify your choice, backing it up with empirical evidence. However remember that an academic essay is not a murder mystery: you should already have briefly discussed your recommendation in the introduction and in other parts of the essay. Do not save it for a grand reveal at the end.

\section{Conclusion}

Write your conclusion here. The conclusion should do more than summarise the essay, making clear the contribution of the work and highlighting key points, limitations, and outstanding questions. It should not introduce any new content or information.

\bibliographystyle{ieeetran}
\bibliography{references}

\end{document}
